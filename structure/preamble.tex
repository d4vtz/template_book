% ========================================
% PREAMBLE.TEX - Configuración del documento
% ========================================

\usepackage{silence}
\WarningFilter{latex}{Command}
\usepackage[scaled]{helvet}     % Fuente del documento
\renewcommand{\familydefault}{\sfdefault}

% ========== CODIFICACIÓN Y IDIOMA ==========
\usepackage[utf8]{inputenc}
\usepackage[spanish]{babel}
\usepackage[T1]{fontenc}
\usepackage{csquotes}

% ====== MATEMÁTICAS, FISICA Y QUIMICA ======
\usepackage{amsmath}            % Ecuaciones y entornos matemáticos
\usepackage{amsthm}             % Teoremas y demostraciones
\usepackage{amssymb}            % Símbolos matemáticos
\usepackage{amsfonts}           % Fuentes matemáticas
\usepackage{mathtools}          % Extensión de amsmath
\usepackage{tensor}             % Notación tensorial (física)
%\usepackage{physics}            % Comandos de física (derivadas, vectores, etc.)
\usepackage[version=4]{mhchem}  % Fórmulas químicas y reacciones
\usepackage{chemfig}            % Estructuras moleculares
\usepackage{chemformula}        % Fórmulas químicas alternativas
\usepackage{siunitx}            % Sistema Internacional de Unidades
\sisetup{
    output-decimal-marker = {,},
    separate-uncertainty = true,
    per-mode = symbol
}

% ========== GEOMETRÍA Y FORMATO ==========
\usepackage{geometry}
\geometry{
    a4paper,
    margin=2.5cm,
    top=3cm,
    bottom=3cm
}
\usepackage{fancyhdr}       % Encabezados y pies de página
\usepackage{setspace}       % Espaciado entre líneas

% ========== GRÁFICOS Y FIGURAS ==========
\usepackage{graphicx}       % Incluir imágenes
\usepackage{float}          % Control de posición de figuras
\usepackage{caption}        % Personalizar captions
\usepackage{subcaption}     % Subfiguras
\usepackage{tikz}           % Dibujos y diagramas
\usepackage{anyfontsize}
\usepackage{pgfplots}       % Gráficas matemáticas
\usepackage{circuitikz}     % Circuitos eléctricos
\pgfplotsset{compat=1.18}

% ========== TABLAS ==========
\usepackage{array}          % Mejorar tablas
\usepackage{booktabs}       % Líneas profesionales en tablas
\usepackage{multirow}       % Celdas que abarcan múltiples filas
\usepackage{longtable}      % Tablas largas

% ========== LISTAS Y ENUMERACIONES ==========
\usepackage{enumerate}      % Listas personalizadas
\usepackage{enumitem}       % Control avanzado de listas

% ========== CÓDIGO Y ALGORITMOS ==========
\usepackage{listings}       % Código fuente
\usepackage{algorithm}      % Algoritmos
\usepackage{algorithmic}    % Pseudocódigo

% ========== COLORES ==========
\usepackage{xcolor}
\definecolor{myblue}{RGB}{0,82,155}
\definecolor{mygreen}{RGB}{0,128,0}
\definecolor{myred}{RGB}{200,0,0}

% ========== HIPERVÍNCULOS ==========
\usepackage{hyperref}
\hypersetup{
    colorlinks=true,
    linkcolor=myblue,
    filecolor=myred,
    urlcolor=myblue,
    citecolor=mygreen,
    pdftitle={Notas de Clase},
    pdfauthor={Tu Nombre},
}
\usepackage{cleveref}       % Referencias inteligentes

% ========== BIBLIOGRAFÍA ==========
% ========================================
% BIBLIOGRAPHY.TEX
% Configuración de bibliografía
% ========================================

\usepackage[backend=biber,style=numeric,sorting=none]{biblatex}

% Agregar archivos de bibliografía
\addbibresource{bibliography/references.bib}

% Configuración adicional de biblatex
\DefineBibliographyStrings{spanish}{
    bibliography = {Referencias},
    references = {Referencias},
}



% ============= ESTILOS ============
\usepackage{structure/styles/style}
\usepackage{structure/styles/environments}

% ============= portada ============
% ========================================
% PORTADA - Estilo Geométrico
% ========================================
\usetikzlibrary{ shapes.geometric }
\usetikzlibrary{calc}
\newcommand{\portada}[3]{
    \pagestyle{empty}
    \begin{tikzpicture}[remember picture,overlay]
        %%%%%%%%%%%%%%%%%%%% Background %%%%%%%%%%%%%%%%%%%%%%%%
        \fill[primary] (current page.south west) rectangle (current page.north east);


        \foreach \i in {2.5,...,22}
            {
                \node[rounded corners,primary!60,draw,regular polygon,regular polygon sides=6, minimum size=\i cm,ultra thick] at ($(current page.west)+(2.5,-5)$) {} ;
            }

        %%%%%%%%%%%%%%%%%%%% Background Polygon %%%%%%%%%%%%%%%%%%%% 
        \foreach \i in {0.5,...,22}
            {
                \node[rounded corners,primary!60,draw,regular polygon,regular polygon sides=6, minimum size=\i cm,ultra thick] at ($(current page.north west)+(2.5,0)$) {} ;
            }

        \foreach \i in {0.5,...,22}
            {
                \node[rounded corners,primary!90,draw,regular polygon,regular polygon sides=6, minimum size=\i cm,ultra thick] at ($(current page.north east)+(0,-9.5)$) {} ;
            }


        \foreach \i in {21,...,6}
            {
                \node[primary!85,rounded corners,draw,regular polygon,regular polygon sides=6, minimum size=\i cm,ultra thick] at ($(current page.south east)+(-0.2,-0.45)$) {} ;
            }


        %%%%%%%%%%%%%%%%%%%% Title of the Report %%%%%%%%%%%%%%%%%%%% 
        \node[left,primary!5,minimum width=0.625*\paperwidth,minimum height=3cm, rounded corners] at ($(current page.north east)+(0,-9.5)$)
        {
            {\fontsize{25}{30} \selectfont \bfseries #1}
        };

        %%%%%%%%%%%%%%%%%%%% Subtitle %%%%%%%%%%%%%%%%%%%% 
        \node[left,primary!10,minimum width=0.625*\paperwidth,minimum height=2cm, rounded corners] at ($(current page.north east)+(0,-11)$)
        {
            {\huge \textit{#2}}
        };

        %%%%%%%%%%%%%%%%%%%% Author Name %%%%%%%%%%%%%%%%%%%% 
        \node[left,primary!5,minimum width=0.625*\paperwidth,minimum height=2cm, rounded corners] at ($(current page.north east)+(0,-13)$)
        {
            {\Large \textsc{#3}}
        };

        %%%%%%%%%%%%%%%%%%%% Year %%%%%%%%%%%%%%%%%%%% 
        \node[rounded corners,fill=primary!70,text =primary!5,regular polygon,regular polygon sides=6, minimum size=2.5 cm,inner sep=0,ultra thick] at ($(current page.west)+(2.5,-5)$) {\LARGE \bfseries \the\year{}};

    \end{tikzpicture}
}